% Options for packages loaded elsewhere
% Options for packages loaded elsewhere
\PassOptionsToPackage{unicode}{hyperref}
\PassOptionsToPackage{hyphens}{url}
\PassOptionsToPackage{dvipsnames,svgnames,x11names}{xcolor}
%
\documentclass[
  10pt,
  letterpaper]{article}
\usepackage{xcolor}
\usepackage{amsmath,amssymb}
\setcounter{secnumdepth}{-\maxdimen} % remove section numbering
\usepackage{iftex}
\ifPDFTeX
  \usepackage[T1]{fontenc}
  \usepackage[utf8]{inputenc}
  \usepackage{textcomp} % provide euro and other symbols
\else % if luatex or xetex
  \usepackage{unicode-math} % this also loads fontspec
  \defaultfontfeatures{Scale=MatchLowercase}
  \defaultfontfeatures[\rmfamily]{Ligatures=TeX,Scale=1}
\fi
\usepackage{lmodern}
\ifPDFTeX\else
  % xetex/luatex font selection
\fi
% Use upquote if available, for straight quotes in verbatim environments
\IfFileExists{upquote.sty}{\usepackage{upquote}}{}
\IfFileExists{microtype.sty}{% use microtype if available
  \usepackage[]{microtype}
  \UseMicrotypeSet[protrusion]{basicmath} % disable protrusion for tt fonts
}{}
\makeatletter
\@ifundefined{KOMAClassName}{% if non-KOMA class
  \IfFileExists{parskip.sty}{%
    \usepackage{parskip}
  }{% else
    \setlength{\parindent}{0pt}
    \setlength{\parskip}{6pt plus 2pt minus 1pt}}
}{% if KOMA class
  \KOMAoptions{parskip=half}}
\makeatother
% Make \paragraph and \subparagraph free-standing
\makeatletter
\ifx\paragraph\undefined\else
  \let\oldparagraph\paragraph
  \renewcommand{\paragraph}{
    \@ifstar
      \xxxParagraphStar
      \xxxParagraphNoStar
  }
  \newcommand{\xxxParagraphStar}[1]{\oldparagraph*{#1}\mbox{}}
  \newcommand{\xxxParagraphNoStar}[1]{\oldparagraph{#1}\mbox{}}
\fi
\ifx\subparagraph\undefined\else
  \let\oldsubparagraph\subparagraph
  \renewcommand{\subparagraph}{
    \@ifstar
      \xxxSubParagraphStar
      \xxxSubParagraphNoStar
  }
  \newcommand{\xxxSubParagraphStar}[1]{\oldsubparagraph*{#1}\mbox{}}
  \newcommand{\xxxSubParagraphNoStar}[1]{\oldsubparagraph{#1}\mbox{}}
\fi
\makeatother


\usepackage{longtable,booktabs,array}
\usepackage{calc} % for calculating minipage widths
% Correct order of tables after \paragraph or \subparagraph
\usepackage{etoolbox}
\makeatletter
\patchcmd\longtable{\par}{\if@noskipsec\mbox{}\fi\par}{}{}
\makeatother
% Allow footnotes in longtable head/foot
\IfFileExists{footnotehyper.sty}{\usepackage{footnotehyper}}{\usepackage{footnote}}
\makesavenoteenv{longtable}
\usepackage{graphicx}
\makeatletter
\newsavebox\pandoc@box
\newcommand*\pandocbounded[1]{% scales image to fit in text height/width
  \sbox\pandoc@box{#1}%
  \Gscale@div\@tempa{\textheight}{\dimexpr\ht\pandoc@box+\dp\pandoc@box\relax}%
  \Gscale@div\@tempb{\linewidth}{\wd\pandoc@box}%
  \ifdim\@tempb\p@<\@tempa\p@\let\@tempa\@tempb\fi% select the smaller of both
  \ifdim\@tempa\p@<\p@\scalebox{\@tempa}{\usebox\pandoc@box}%
  \else\usebox{\pandoc@box}%
  \fi%
}
% Set default figure placement to htbp
\def\fps@figure{htbp}
\makeatother





\setlength{\emergencystretch}{3em} % prevent overfull lines

\providecommand{\tightlist}{%
  \setlength{\itemsep}{0pt}\setlength{\parskip}{0pt}}



 
\usepackage[numbers]{natbib}
\bibliographystyle{plainnat}


\usepackage{cogsci}
\usepackage{pslatex}
\makeatletter
\@ifpackageloaded{caption}{}{\usepackage{caption}}
\AtBeginDocument{%
\ifdefined\contentsname
  \renewcommand*\contentsname{Table of contents}
\else
  \newcommand\contentsname{Table of contents}
\fi
\ifdefined\listfigurename
  \renewcommand*\listfigurename{List of Figures}
\else
  \newcommand\listfigurename{List of Figures}
\fi
\ifdefined\listtablename
  \renewcommand*\listtablename{List of Tables}
\else
  \newcommand\listtablename{List of Tables}
\fi
\ifdefined\figurename
  \renewcommand*\figurename{Figure}
\else
  \newcommand\figurename{Figure}
\fi
\ifdefined\tablename
  \renewcommand*\tablename{Table}
\else
  \newcommand\tablename{Table}
\fi
}
\@ifpackageloaded{float}{}{\usepackage{float}}
\floatstyle{ruled}
\@ifundefined{c@chapter}{\newfloat{codelisting}{h}{lop}}{\newfloat{codelisting}{h}{lop}[chapter]}
\floatname{codelisting}{Listing}
\newcommand*\listoflistings{\listof{codelisting}{List of Listings}}
\makeatother
\makeatletter
\makeatother
\makeatletter
\@ifpackageloaded{caption}{}{\usepackage{caption}}
\@ifpackageloaded{subcaption}{}{\usepackage{subcaption}}
\makeatother
\usepackage{bookmark}
\IfFileExists{xurl.sty}{\usepackage{xurl}}{} % add URL line breaks if available
\urlstyle{same}
\hypersetup{
  pdftitle={Procedural Learning with Graded Entropy},
  colorlinks=true,
  linkcolor={blue},
  filecolor={Maroon},
  citecolor={Blue},
  urlcolor={Blue},
  pdfcreator={LaTeX via pandoc}}


\title{Procedural Learning with Graded Entropy}
\author{}
\date{}
\begin{document}
\author{{\large \bf Cherrie Chang (cchang@mghihp.edu)} \\
  Advanced Analytics \& AI for Communication Science Group, Mass General Hospital Institute of Health Professions,\\Charlestown, MA, USA
  \AND {\large \bf Tianyi Li (lidzr@bc.edu)} \\
  Department of Psychology and Neuroscience, Boston College, Chestnut Hill, MA, USA
  \AND {\large \bf Joshua Hartshorne (jharts@example.edu)} \\
  Advanced Analytics \& AI for Communication Science Group, Mass General Hospital Institute of Health Professions, \\Charlestown, MA, USA}
\maketitle


\section{Abstract}\label{abstract}

Your abstract text here. The abstract should be one paragraph. Following
the abstract should be keywords.

\textbf{Keywords:} procedural learning, implicit learning, statistical
learning

\section{Introduction}\label{introduction}

Your introduction text here. You can cite references using
\texttt{@NewellSimon1972a} or \texttt{{[}@ChalnickBillman1988a{]}}.

\section{Method}\label{method}

\subsection{Participants}\label{participants}

We recruited 251 participants (133 female, 118 male) on the online
recruitment platform Prolific, screening for participants with an
approval rating above 95\% from previous studies. Participants ranged in
age from 18 to 75 years (\(\mu\)=35.6, Mdn=33, \(\sigma\)=11.8748024),
representing 32 nationalities and 20 primary languages. Participants
were compensated at \$12.00 per hour, with expected completion time
varying by condition (\textasciitilde20-40 minutes). All participants
provided informed consent prior to beginning the experiment. The study
was approved by the Institutional Review Board at the MGH Institute of
Health Professions.

\subsection{Probabilistic Serial Reaction Time
Task}\label{probabilistic-serial-reaction-time-task}

\phantomsection\label{example-trial-screen}
\begin{figure}[H]

{\centering \includegraphics[width=0.8\linewidth,height=\textheight,keepaspectratio]{"../assets/example-trial-screen.png"}

}

\caption{An example trial screen in the 8-position condition of the
experiment}

\end{figure}%

Like other serial reaction time task designs, our study tasks
participants to respond to a visual stimulus that appears in one of
several evenly-spaced positions on screen as quickly and accurately as
possible, by pressing a corresponding key on the keyboard. In our
implementation, this visual stimulus is a mole clad in nothing but a
bright red bib and matching sunglasses (Figure 1); and a schematic of
each position-to-keyboard mapping for each condition is shown in (Table
1). Participants were evenly divided into 5 groups of 50 per condition,
with the conditions differing in the number of positions (4, 5, 6, 7, 8)
the mole can appear in.

A participant is first given a text-based tutorial accompanied by an
animated demonstration on which key to press in response to each
position, then instructed to work through a set of practice trials where
each position is visited twice in random order. After the practice
section, the participant moves on to complete 20 blocks of trials, each
separated with a self-paced break. Each block consists of trials 10x the
number of positions in the participant's assigned condition. This design
choice results in longer blocks for conditions with more positions, but
ensures each position is visited an equal number of times on average
across conditions. In both practice and main trials, participants are
given feedback on correctness via a short message (a checkmark vs.~``Try
again!'' + an error tone), and allowed to retry each failed trial until
they respond correctly. Following standard SRTT protocol, there is also
a 120ms response-stimulus interval (RSI) between trials {[}TODO:
cite{]}. We find during piloting that allowing retries, combined with
the 120ms RSI, prevents participants from making compensatory errors,
where they too quickly press the wrong key in the next trial in an
attempt to correct their current trial. After each block, participants
are given adaptive feedback based on their accuracy and speed. At the
end of the experiment, participants complete a brief questionnaire
probing their explicit awareness of any patterns in the mole's
appearances (Table 2).

In addition to our well-dressed mole, a number of design choices were
made to facilitate participant understanding and engagement. During the
practice phase, each position on screen is labelled with its
corresponding key character (e.g.~``A'', ``S'', ``D'') to help
familiarize the participant with the keyboard mappings. These key labels
disappear in the main trials to prevent participants from explicitly
encoding the sequence of positions via their character labels, but
reappear during retry of failed trials. To make the correspondence
between the positions on screen and keyboard keys as automatic and
intuitive as possible, we chose to use conventional QWERTY-based finger
layouts for the key mappings across conditions (Table 2). While this
decision helped reduce cognitive load from learning novel finger
placements for most participants, we found during piloting that the
larger gap between keys ``F'' and ``J'' on the keyboard introduces a
discrepancy between the evenly-spaced on-screen positions and the
unevenly spaced keyboard keys. This made participants more prone to
making errors in the positions towards the middle of the screen compared
to those on the outer edges, especially when number of positions scales
{[}TODO: maybe run stats on pilot to find this effect{]}; and was not
mitigated by shifting either hand's placement to close the extra gap,
because the positions were still corresponding half to one hand and half
to the other. After further piloting testing different visual aids, we
settled on adding two hands visually to the bottom of the screen, each
finger aligned with its target position (Figure 1). We received verbal
feedback that this visual aid helped participants better map the spatial
layout of the on-screen positions to their corresponding keys. Finally,
the visual design of the ``positions'' on screen mimicked blank 3D
keyboard keys that light up in pink when the mole appears on top, and
look visibly ``pressed down'' when the participant presses the
corresponding keyboard key. This design choice sought to make the
on-screen positions diegetically linked to the keyboard keys to
maximally reduce the mental distance between the two. This is important
to enforce as much learning via motor memory as possible, rather than
via visual-spatial memory of the on-screen positions.

\subsection{Transition Matrices}\label{transition-matrices}

Within each block, the sequence of positions is generated based on a
transition matrix specific to the participant's assigned condition (see
Transition Matrices section below for details). The sequence of
positions is generated based on transition matrices of varying sizes,
which define the probabilities of transitioning from one position to
another.

Our experiment involves a set of keyuses transition matrices to generate
the sequence of positions in which a mole appears. Each matrix defines
the probabilities of transitioning from one position to another on the
screen, with the size of the matrix corresponding to the number of
positions. For example, a 4x4 matrix defines transitions among 4
positions, while an 8x8 matrix defines transitions among 8 positions.

Between-subjects manipulation of the number of positions to handle,
which is the same thing as the size of the transition matrix used to
generate the position sequence. Participants are randomly assigned to
one of five matrix size conditions (4x4, 5x5, 6x6, 7x7, 8x8). The
transition matrix describes the number of positions on screen to respond
to, and the probabilistic structure the transitions between positions
follow. Because larger matrix sizes/sequences with more positions
require more trials for the participant to observe the same number of
trials on average for each position compared to smaller matrix sizes, we
also vary the total number of trials across conditions, setting it as 20
blocks * 10 * matrix\_length for each condition. We also vary the total
number of practice trials similarly, with each condition getting 2 *
matrix\_length number of practice trials.

\subsection{Hypotheses}\label{hypotheses}

\subsection{Matrix Construction and
Verification}\label{matrix-construction-and-verification}

\section{Method}\label{method-1}

\subsection{Participants}\label{participants-1}

\subsection{Protocol}\label{protocol}

\subsection{Materials}\label{materials}

\section{Results}\label{results}

\subsection{H0}\label{h0}

\subsection{H1}\label{h1}

\subsection{H2}\label{h2}

\subsection{H3}\label{h3}

\section{Discussion}\label{discussion}

\section{References}\label{references}

\renewcommand{\bibsection}{}
\bibliography{CogSci_Template.bib}





\end{document}
