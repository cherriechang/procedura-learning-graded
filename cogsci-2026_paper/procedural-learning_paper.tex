% Options for packages loaded elsewhere
% Options for packages loaded elsewhere
\PassOptionsToPackage{unicode}{hyperref}
\PassOptionsToPackage{hyphens}{url}
\PassOptionsToPackage{dvipsnames,svgnames,x11names}{xcolor}
%
\documentclass[
  10pt,
  letterpaper]{article}
\usepackage{xcolor}
\usepackage{amsmath,amssymb}
\setcounter{secnumdepth}{-\maxdimen} % remove section numbering
\usepackage{iftex}
\ifPDFTeX
  \usepackage[T1]{fontenc}
  \usepackage[utf8]{inputenc}
  \usepackage{textcomp} % provide euro and other symbols
\else % if luatex or xetex
  \usepackage{unicode-math} % this also loads fontspec
  \defaultfontfeatures{Scale=MatchLowercase}
  \defaultfontfeatures[\rmfamily]{Ligatures=TeX,Scale=1}
\fi
\usepackage{lmodern}
\ifPDFTeX\else
  % xetex/luatex font selection
\fi
% Use upquote if available, for straight quotes in verbatim environments
\IfFileExists{upquote.sty}{\usepackage{upquote}}{}
\IfFileExists{microtype.sty}{% use microtype if available
  \usepackage[]{microtype}
  \UseMicrotypeSet[protrusion]{basicmath} % disable protrusion for tt fonts
}{}
\makeatletter
\@ifundefined{KOMAClassName}{% if non-KOMA class
  \IfFileExists{parskip.sty}{%
    \usepackage{parskip}
  }{% else
    \setlength{\parindent}{0pt}
    \setlength{\parskip}{6pt plus 2pt minus 1pt}}
}{% if KOMA class
  \KOMAoptions{parskip=half}}
\makeatother
% Make \paragraph and \subparagraph free-standing
\makeatletter
\ifx\paragraph\undefined\else
  \let\oldparagraph\paragraph
  \renewcommand{\paragraph}{
    \@ifstar
      \xxxParagraphStar
      \xxxParagraphNoStar
  }
  \newcommand{\xxxParagraphStar}[1]{\oldparagraph*{#1}\mbox{}}
  \newcommand{\xxxParagraphNoStar}[1]{\oldparagraph{#1}\mbox{}}
\fi
\ifx\subparagraph\undefined\else
  \let\oldsubparagraph\subparagraph
  \renewcommand{\subparagraph}{
    \@ifstar
      \xxxSubParagraphStar
      \xxxSubParagraphNoStar
  }
  \newcommand{\xxxSubParagraphStar}[1]{\oldsubparagraph*{#1}\mbox{}}
  \newcommand{\xxxSubParagraphNoStar}[1]{\oldsubparagraph{#1}\mbox{}}
\fi
\makeatother


\usepackage{longtable,booktabs,array}
\usepackage{calc} % for calculating minipage widths
% Correct order of tables after \paragraph or \subparagraph
\usepackage{etoolbox}
\makeatletter
\patchcmd\longtable{\par}{\if@noskipsec\mbox{}\fi\par}{}{}
\makeatother
% Allow footnotes in longtable head/foot
\IfFileExists{footnotehyper.sty}{\usepackage{footnotehyper}}{\usepackage{footnote}}
\makesavenoteenv{longtable}
\usepackage{graphicx}
\makeatletter
\newsavebox\pandoc@box
\newcommand*\pandocbounded[1]{% scales image to fit in text height/width
  \sbox\pandoc@box{#1}%
  \Gscale@div\@tempa{\textheight}{\dimexpr\ht\pandoc@box+\dp\pandoc@box\relax}%
  \Gscale@div\@tempb{\linewidth}{\wd\pandoc@box}%
  \ifdim\@tempb\p@<\@tempa\p@\let\@tempa\@tempb\fi% select the smaller of both
  \ifdim\@tempa\p@<\p@\scalebox{\@tempa}{\usebox\pandoc@box}%
  \else\usebox{\pandoc@box}%
  \fi%
}
% Set default figure placement to htbp
\def\fps@figure{htbp}
\makeatother





\setlength{\emergencystretch}{3em} % prevent overfull lines

\providecommand{\tightlist}{%
  \setlength{\itemsep}{0pt}\setlength{\parskip}{0pt}}



 
\usepackage[numbers]{natbib}
\bibliographystyle{plainnat}


\usepackage{cogsci}
\usepackage{pslatex}
\makeatletter
\@ifpackageloaded{caption}{}{\usepackage{caption}}
\AtBeginDocument{%
\ifdefined\contentsname
  \renewcommand*\contentsname{Table of contents}
\else
  \newcommand\contentsname{Table of contents}
\fi
\ifdefined\listfigurename
  \renewcommand*\listfigurename{List of Figures}
\else
  \newcommand\listfigurename{List of Figures}
\fi
\ifdefined\listtablename
  \renewcommand*\listtablename{List of Tables}
\else
  \newcommand\listtablename{List of Tables}
\fi
\ifdefined\figurename
  \renewcommand*\figurename{Figure}
\else
  \newcommand\figurename{Figure}
\fi
\ifdefined\tablename
  \renewcommand*\tablename{Table}
\else
  \newcommand\tablename{Table}
\fi
}
\@ifpackageloaded{float}{}{\usepackage{float}}
\floatstyle{ruled}
\@ifundefined{c@chapter}{\newfloat{codelisting}{h}{lop}}{\newfloat{codelisting}{h}{lop}[chapter]}
\floatname{codelisting}{Listing}
\newcommand*\listoflistings{\listof{codelisting}{List of Listings}}
\makeatother
\makeatletter
\makeatother
\makeatletter
\@ifpackageloaded{caption}{}{\usepackage{caption}}
\@ifpackageloaded{subcaption}{}{\usepackage{subcaption}}
\makeatother
\usepackage{bookmark}
\IfFileExists{xurl.sty}{\usepackage{xurl}}{} % add URL line breaks if available
\urlstyle{same}
\hypersetup{
  pdftitle={Procedural Learning with Graded Entropy},
  colorlinks=true,
  linkcolor={blue},
  filecolor={Maroon},
  citecolor={Blue},
  urlcolor={Blue},
  pdfcreator={LaTeX via pandoc}}


\title{Procedural Learning with Graded Entropy}
\author{}
\date{}
\begin{document}
\author{{\large \bf Cherrie Chang (cchang@mghihp.edu)} \\
  Advanced Analytics \& AI for Communication Science Group, Mass General Hospital Institute of Health Professions,\\Charlestown, MA, USA
  \AND {\large \bf Tianyi Li (lidzr@bc.edu)} \\
  Department of Psychology and Neuroscience, Boston College, Chestnut Hill, MA, USA
  \AND {\large \bf Joshua Hartshorne (jharts@example.edu)} \\
  Advanced Analytics \& AI for Communication Science Group, Mass General Hospital Institute of Health Professions, \\Charlestown, MA, USA}
\maketitle


\section{Abstract}\label{abstract}

Your abstract text here. The abstract should be one paragraph. Following
the abstract should be keywords.

\textbf{Keywords:} procedural learning, implicit learning, statistical
learning

\section{Introduction}\label{introduction}

Your introduction text here. You can cite references using
\texttt{@NewellSimon1972a} or \texttt{{[}@ChalnickBillman1988a{]}}.

\section{Method}\label{method}

\subsection{Overall Design}\label{overall-design}

Our experiment involves a set of keyuses transition matrices to generate
the sequence of positions in which a mole appears. Each matrix defines
the probabilities of transitioning from one position to another on the
screen, with the size of the matrix corresponding to the number of
positions. For example, a 4x4 matrix defines transitions among 4
positions, while an 8x8 matrix defines transitions among 8 positions.

\subsection{Participants}\label{participants}

We recruited 251 participants (133 female, 118 male) on the online
recruitment platform Prolific, with 50 participants per condition (51 in
5x5 condition). We screened for participants with an approval rating
above 95\% from previous Prolific studies. Participants ranged in age
from 18 to 75 years (\(\mu\)=35.6, Mdn=33, \(\sigma\)=11.8748024),
spanning 32 nationalities and 20 languages spoken. Participants were
compensated at a rate of \$12.00 per hour, with the experiment expected
to take \textasciitilde20 (4x4 condition) to \textasciitilde40 (5x5
condition) minutes in total. Partipants provided informed consent prior
to beginning the experiment and all procedures have been approved by the
Institutional Review Board at the MGH Institute of Health Professions.

\subsection{Transition Matrices}\label{transition-matrices}

We between-subjects manipulation of the number of positions to handle,
which is the same thing as the size of the transition matrix used to
generate the position sequence. Participants are randomly assigned to
one of five matrix size conditions (4x4, 5x5, 6x6, 7x7, 8x8). The
transition matrix describes the number of positions on screen to respond
to, and the probabilistic structure the transitions between positions
follow. Because larger matrix sizes/sequences with more positions
require more trials for the participant to observe the same number of
trials on average for each position compared to smaller matrix sizes, we
also vary the total number of trials across conditions, setting it as 20
blocks * 10 * matrix\_length for each condition. We also vary the total
number of practice trials similarly, with each condition getting 2 *
matrix\_length number of practice trials.

\subsection{Hypotheses}\label{hypotheses}

\subsection{Matrix Construction and
Verification}\label{matrix-construction-and-verification}

\section{Method}\label{method-1}

\subsection{Participants}\label{participants-1}

\subsection{Protocol}\label{protocol}

\subsection{Materials}\label{materials}

\section{Results}\label{results}

\subsection{H0}\label{h0}

\subsection{H1}\label{h1}

\subsection{H2}\label{h2}

\subsection{H3}\label{h3}

\section{Discussion}\label{discussion}

\section{References}\label{references}

\renewcommand{\bibsection}{}
\bibliography{CogSci_Template.bib}





\end{document}
